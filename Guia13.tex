\documentclass[10pt,a4paper]{article}\usepackage[]{graphicx}\usepackage[]{color}
%% maxwidth is the original width if it is less than linewidth
%% otherwise use linewidth (to make sure the graphics do not exceed the margin)
\makeatletter
\def\maxwidth{ %
  \ifdim\Gin@nat@width>\linewidth
    \linewidth
  \else
    \Gin@nat@width
  \fi
}
\makeatother

\definecolor{fgcolor}{rgb}{0.345, 0.345, 0.345}
\newcommand{\hlnum}[1]{\textcolor[rgb]{0.686,0.059,0.569}{#1}}%
\newcommand{\hlstr}[1]{\textcolor[rgb]{0.192,0.494,0.8}{#1}}%
\newcommand{\hlcom}[1]{\textcolor[rgb]{0.678,0.584,0.686}{\textit{#1}}}%
\newcommand{\hlopt}[1]{\textcolor[rgb]{0,0,0}{#1}}%
\newcommand{\hlstd}[1]{\textcolor[rgb]{0.345,0.345,0.345}{#1}}%
\newcommand{\hlkwa}[1]{\textcolor[rgb]{0.161,0.373,0.58}{\textbf{#1}}}%
\newcommand{\hlkwb}[1]{\textcolor[rgb]{0.69,0.353,0.396}{#1}}%
\newcommand{\hlkwc}[1]{\textcolor[rgb]{0.333,0.667,0.333}{#1}}%
\newcommand{\hlkwd}[1]{\textcolor[rgb]{0.737,0.353,0.396}{\textbf{#1}}}%

\usepackage{framed}
\makeatletter
\newenvironment{kframe}{%
 \def\at@end@of@kframe{}%
 \ifinner\ifhmode%
  \def\at@end@of@kframe{\end{minipage}}%
  \begin{minipage}{\columnwidth}%
 \fi\fi%
 \def\FrameCommand##1{\hskip\@totalleftmargin \hskip-\fboxsep
 \colorbox{shadecolor}{##1}\hskip-\fboxsep
     % There is no \\@totalrightmargin, so:
     \hskip-\linewidth \hskip-\@totalleftmargin \hskip\columnwidth}%
 \MakeFramed {\advance\hsize-\width
   \@totalleftmargin\z@ \linewidth\hsize
   \@setminipage}}%
 {\par\unskip\endMakeFramed%
 \at@end@of@kframe}
\makeatother

\definecolor{shadecolor}{rgb}{.97, .97, .97}
\definecolor{messagecolor}{rgb}{0, 0, 0}
\definecolor{warningcolor}{rgb}{1, 0, 1}
\definecolor{errorcolor}{rgb}{1, 0, 0}
\newenvironment{knitrout}{}{} % an empty environment to be redefined in TeX

\usepackage{alltt}
\usepackage[latin1]{inputenc}
\usepackage{amsmath}
\usepackage{amsfonts}
\usepackage{amssymb}
\author{Erika Martínez}
\title{Guías prácticas}
\IfFileExists{upquote.sty}{\usepackage{upquote}}{}
\begin{document}

\maketitle
\newpage

UNIDAD 3: Practica 13 - Espacios muestrales

GENERACION DE ESPACIOS MUESTRALES Y DE MUESTRAS ALEATORIAS.

Simular 10 lanzamientos de una moneda 
\begin{knitrout}
\definecolor{shadecolor}{rgb}{0.969, 0.969, 0.969}\color{fgcolor}\begin{kframe}
\begin{alltt}
\hlcom{# vector del cual se tomar? la muestra }
\hlstd{moneda} \hlkwb{<-} \hlkwd{c}\hlstd{(}\hlstr{"C"}\hlstd{,} \hlstr{"+"}\hlstd{); moneda}
\end{alltt}
\begin{verbatim}
## [1] "C" "+"
\end{verbatim}
\begin{alltt}
\hlcom{# tama?o de la muestra }
\hlstd{n} \hlkwb{<-} \hlnum{10}\hlstd{; n}
\end{alltt}
\begin{verbatim}
## [1] 10
\end{verbatim}
\begin{alltt}
\hlcom{#generando la muestra aleatoria con reemplazamiento}
\hlstd{lanzamientos} \hlkwb{<-} \hlkwd{sample}\hlstd{(moneda, n,} \hlkwc{replace}\hlstd{=}\hlnum{TRUE}\hlstd{); lanzamientos}
\end{alltt}
\begin{verbatim}
##  [1] "C" "C" "C" "C" "+" "C" "+" "C" "+" "+"
\end{verbatim}
\end{kframe}
\end{knitrout}

Elegir 6 n?meros de una loter?a de 54 n?meros 
\begin{knitrout}
\definecolor{shadecolor}{rgb}{0.969, 0.969, 0.969}\color{fgcolor}\begin{kframe}
\begin{alltt}
\hlcom{# se define el espacio muestral del cual se tomar? la muestra}
\hlstd{espacio} \hlkwb{<-} \hlnum{1}\hlopt{:}\hlnum{54}\hlstd{;espacio}
\end{alltt}
\begin{verbatim}
##  [1]  1  2  3  4  5  6  7  8  9 10 11 12 13 14 15 16 17 18 19 20 21 22 23
## [24] 24 25 26 27 28 29 30 31 32 33 34 35 36 37 38 39 40 41 42 43 44 45 46
## [47] 47 48 49 50 51 52 53 54
\end{verbatim}
\begin{alltt}
\hlcom{# se define el tama?o de la muestra }
\hlstd{n} \hlkwb{<-} \hlnum{6}\hlstd{; n}
\end{alltt}
\begin{verbatim}
## [1] 6
\end{verbatim}
\begin{alltt}
\hlcom{#seleccionando la muestra sin reposici?n }
\hlstd{muestra} \hlkwb{<-} \hlkwd{sample}\hlstd{(espacio, n); muestra}
\end{alltt}
\begin{verbatim}
## [1] 26 38 47 51 41 23
\end{verbatim}
\end{kframe}
\end{knitrout}

Simular 4 lanzamientos de dos dados 
\begin{knitrout}
\definecolor{shadecolor}{rgb}{0.969, 0.969, 0.969}\color{fgcolor}\begin{kframe}
\begin{alltt}
\hlstd{espacio} \hlkwb{=} \hlkwd{as.vector}\hlstd{(}\hlkwd{outer}\hlstd{(}\hlnum{1}\hlopt{:}\hlnum{6}\hlstd{,} \hlnum{1}\hlopt{:}\hlnum{6}\hlstd{, paste)); espacio}
\end{alltt}
\begin{verbatim}
##  [1] "1 1" "2 1" "3 1" "4 1" "5 1" "6 1" "1 2" "2 2" "3 2" "4 2" "5 2"
## [12] "6 2" "1 3" "2 3" "3 3" "4 3" "5 3" "6 3" "1 4" "2 4" "3 4" "4 4"
## [23] "5 4" "6 4" "1 5" "2 5" "3 5" "4 5" "5 5" "6 5" "1 6" "2 6" "3 6"
## [34] "4 6" "5 6" "6 6"
\end{verbatim}
\begin{alltt}
\hlcom{# se define el tama?o de la muestra }
\hlstd{n} \hlkwb{<-} \hlnum{4}\hlstd{; n}
\end{alltt}
\begin{verbatim}
## [1] 4
\end{verbatim}
\begin{alltt}
\hlcom{# finalmente se selecciona la muestra }
\hlstd{muestra} \hlkwb{<-} \hlkwd{sample}\hlstd{(espacio, n,} \hlkwc{replace}\hlstd{=}\hlnum{TRUE}\hlstd{); muestra}
\end{alltt}
\begin{verbatim}
## [1] "6 2" "5 3" "2 2" "5 3"
\end{verbatim}
\end{kframe}
\end{knitrout}

Seleccionar cinco cartas de un naipe de 52 cartas 
\begin{knitrout}
\definecolor{shadecolor}{rgb}{0.969, 0.969, 0.969}\color{fgcolor}\begin{kframe}
\begin{alltt}
\hlstd{naipe} \hlkwb{=} \hlkwd{paste}\hlstd{(}\hlkwd{rep}\hlstd{(}\hlkwd{c}\hlstd{(}\hlstr{"A"}\hlstd{,} \hlnum{2}\hlopt{:}\hlnum{10}\hlstd{,} \hlstr{"J"}\hlstd{,} \hlstr{"Q"}\hlstd{,} \hlstr{"K"}\hlstd{),} \hlnum{4}\hlstd{),}
\hlkwd{c}\hlstd{(}\hlstr{"OROS"}\hlstd{,}\hlstr{"COPAS"}\hlstd{,} \hlstr{"BASTOS"}\hlstd{,}\hlstr{"ESPADAS"}\hlstd{));naipe}
\end{alltt}
\begin{verbatim}
##  [1] "A OROS"     "2 COPAS"    "3 BASTOS"   "4 ESPADAS"  "5 OROS"    
##  [6] "6 COPAS"    "7 BASTOS"   "8 ESPADAS"  "9 OROS"     "10 COPAS"  
## [11] "J BASTOS"   "Q ESPADAS"  "K OROS"     "A COPAS"    "2 BASTOS"  
## [16] "3 ESPADAS"  "4 OROS"     "5 COPAS"    "6 BASTOS"   "7 ESPADAS" 
## [21] "8 OROS"     "9 COPAS"    "10 BASTOS"  "J ESPADAS"  "Q OROS"    
## [26] "K COPAS"    "A BASTOS"   "2 ESPADAS"  "3 OROS"     "4 COPAS"   
## [31] "5 BASTOS"   "6 ESPADAS"  "7 OROS"     "8 COPAS"    "9 BASTOS"  
## [36] "10 ESPADAS" "J OROS"     "Q COPAS"    "K BASTOS"   "A ESPADAS" 
## [41] "2 OROS"     "3 COPAS"    "4 BASTOS"   "5 ESPADAS"  "6 OROS"    
## [46] "7 COPAS"    "8 BASTOS"   "9 ESPADAS"  "10 OROS"    "J COPAS"   
## [51] "Q BASTOS"   "K ESPADAS"
\end{verbatim}
\begin{alltt}
\hlcom{# se define el tama?o de la muestra }
\hlstd{n} \hlkwb{<-} \hlnum{5}\hlstd{; n}
\end{alltt}
\begin{verbatim}
## [1] 5
\end{verbatim}
\begin{alltt}
\hlcom{# se obtiene la muestra sin reemplazo (aunque no se especifique con replace=FALSE) }
\hlstd{cartas} \hlkwb{<-} \hlkwd{sample}\hlstd{(naipe, n) ; cartas}
\end{alltt}
\begin{verbatim}
## [1] "5 ESPADAS"  "10 ESPADAS" "J COPAS"    "9 BASTOS"   "Q OROS"
\end{verbatim}
\end{kframe}
\end{knitrout}

Generar una muestra aleatoria de tama?o 120,con los n?meros del 1 al 6 en el que las probabilidades de cada uno de los n?meros son respectivamente los siguientes valores: 0.5, 0.25, 0.15, 0.04, 0.03 y 0.003.
\begin{knitrout}
\definecolor{shadecolor}{rgb}{0.969, 0.969, 0.969}\color{fgcolor}\begin{kframe}
\begin{alltt}
\hlkwd{sample}\hlstd{(}\hlnum{1}\hlopt{:}\hlnum{6}\hlstd{,}\hlnum{120}\hlstd{,}\hlkwc{replace}\hlstd{=}\hlnum{TRUE}\hlstd{,} \hlkwd{c}\hlstd{(}\hlnum{0.5}\hlstd{,}\hlnum{0.25}\hlstd{,}\hlnum{0.15}\hlstd{,}\hlnum{0.04}\hlstd{,}\hlnum{0.03}\hlstd{,}\hlnum{0.03}\hlstd{))}
\end{alltt}
\begin{verbatim}
##   [1] 2 3 1 2 1 3 1 2 2 1 6 2 1 1 1 2 1 1 6 5 3 3 2 5 2 1 2 1 1 6 1 4 1 1 1
##  [36] 1 1 2 2 2 3 1 6 2 2 1 1 3 2 1 1 1 3 1 1 3 1 1 2 2 1 2 1 2 1 4 1 2 1 1
##  [71] 2 2 3 2 1 1 1 2 1 1 1 1 1 1 2 3 5 3 6 3 5 3 3 1 1 1 2 3 5 1 1 2 3 5 2
## [106] 1 1 2 2 1 2 1 2 1 2 2 5 3 3 1
\end{verbatim}
\begin{alltt}
\hlcom{#Escriba una funci?n que reciba los n?meros enteros entre 1 y 500 inclusive, la funci?n retornar? el }
\hlcom{#espacio formado por los n?meros divisibles entre 7.Despu?s de llamar a esta funci?n se extraer? }
\hlcom{#aleatoriamente 12 de estos n?meros, con reemplazo.}

\hlcom{# definiendo la funci?n que generar? el espacio formado }
\hlstd{espacio} \hlkwb{<-} \hlkwa{function}\hlstd{(}\hlkwc{num}\hlstd{)}
\hlstd{\{}
 \hlstd{numDiv7} \hlkwb{<-} \hlkwd{numeric}\hlstd{(}\hlnum{0}\hlstd{)}
 \hlstd{ind} \hlkwb{<-} \hlnum{0}
 \hlkwa{for}\hlstd{(i} \hlkwa{in} \hlnum{1}\hlopt{:}\hlkwd{length}\hlstd{(num))}
   \hlkwa{if} \hlstd{((num[i]} \hlopt \hlnum{7}\hlstd{)}\hlopt{==}\hlnum{0}\hlstd{)}
    \hlstd{\{}
     \hlstd{ind} \hlkwb{<-} \hlstd{ind}\hlopt{+}\hlnum{1}
     \hlstd{numDiv7[ind]}\hlkwb{=}\hlstd{num[i]}
    \hlstd{\}}
 \hlkwd{return}\hlstd{(numDiv7)}
\hlstd{\}}
\hlstd{numeros} \hlkwb{<-} \hlnum{1}\hlopt{:}\hlnum{500}
\hlstd{espacio}
\end{alltt}
\begin{verbatim}
## function(num) 
## { 
##  numDiv7 <- numeric(0) 
##  ind <- 0 
##  for(i in 1:length(num)) 
##    if ((num[i] %% 7)==0) 
##     { 
##      ind <- ind+1 
##      numDiv7[ind]=num[i] 
##     } 
##  return(numDiv7) 
## }
\end{verbatim}
\begin{alltt}
\hlcom{# generando el espacio muestral }
\hlstd{s} \hlkwb{<-} \hlkwd{espacio}\hlstd{(numeros); s}
\end{alltt}
\begin{verbatim}
##  [1]   7  14  21  28  35  42  49  56  63  70  77  84  91  98 105 112 119
## [18] 126 133 140 147 154 161 168 175 182 189 196 203 210 217 224 231 238
## [35] 245 252 259 266 273 280 287 294 301 308 315 322 329 336 343 350 357
## [52] 364 371 378 385 392 399 406 413 420 427 434 441 448 455 462 469 476
## [69] 483 490 497
\end{verbatim}
\begin{alltt}
\hlcom{# seleccionando la muestra }
\hlstd{muestra} \hlkwb{<-} \hlkwd{sample}\hlstd{(s,} \hlnum{12}\hlstd{,} \hlkwc{replace}\hlstd{=}\hlnum{TRUE}\hlstd{); muestra}
\end{alltt}
\begin{verbatim}
##  [1] 336 245 273 140  14  14 329 112 161 371 252 294
\end{verbatim}
\end{kframe}
\end{knitrout}

\end{document}
