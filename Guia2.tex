\documentclass[10pt,a4paper]{article}\usepackage[]{graphicx}\usepackage[]{color}
%% maxwidth is the original width if it is less than linewidth
%% otherwise use linewidth (to make sure the graphics do not exceed the margin)
\makeatletter
\def\maxwidth{ %
  \ifdim\Gin@nat@width>\linewidth
    \linewidth
  \else
    \Gin@nat@width
  \fi
}
\makeatother

\definecolor{fgcolor}{rgb}{0.345, 0.345, 0.345}
\newcommand{\hlnum}[1]{\textcolor[rgb]{0.686,0.059,0.569}{#1}}%
\newcommand{\hlstr}[1]{\textcolor[rgb]{0.192,0.494,0.8}{#1}}%
\newcommand{\hlcom}[1]{\textcolor[rgb]{0.678,0.584,0.686}{\textit{#1}}}%
\newcommand{\hlopt}[1]{\textcolor[rgb]{0,0,0}{#1}}%
\newcommand{\hlstd}[1]{\textcolor[rgb]{0.345,0.345,0.345}{#1}}%
\newcommand{\hlkwa}[1]{\textcolor[rgb]{0.161,0.373,0.58}{\textbf{#1}}}%
\newcommand{\hlkwb}[1]{\textcolor[rgb]{0.69,0.353,0.396}{#1}}%
\newcommand{\hlkwc}[1]{\textcolor[rgb]{0.333,0.667,0.333}{#1}}%
\newcommand{\hlkwd}[1]{\textcolor[rgb]{0.737,0.353,0.396}{\textbf{#1}}}%

\usepackage{framed}
\makeatletter
\newenvironment{kframe}{%
 \def\at@end@of@kframe{}%
 \ifinner\ifhmode%
  \def\at@end@of@kframe{\end{minipage}}%
  \begin{minipage}{\columnwidth}%
 \fi\fi%
 \def\FrameCommand##1{\hskip\@totalleftmargin \hskip-\fboxsep
 \colorbox{shadecolor}{##1}\hskip-\fboxsep
     % There is no \\@totalrightmargin, so:
     \hskip-\linewidth \hskip-\@totalleftmargin \hskip\columnwidth}%
 \MakeFramed {\advance\hsize-\width
   \@totalleftmargin\z@ \linewidth\hsize
   \@setminipage}}%
 {\par\unskip\endMakeFramed%
 \at@end@of@kframe}
\makeatother

\definecolor{shadecolor}{rgb}{.97, .97, .97}
\definecolor{messagecolor}{rgb}{0, 0, 0}
\definecolor{warningcolor}{rgb}{1, 0, 1}
\definecolor{errorcolor}{rgb}{1, 0, 0}
\newenvironment{knitrout}{}{} % an empty environment to be redefined in TeX

\usepackage{alltt}
\usepackage[latin1]{inputenc}
\usepackage{amsmath}
\usepackage{amsfonts}
\usepackage{amssymb}
\author{Erika Mart�nez}
\title{Gu�as pr�cticas}
\IfFileExists{upquote.sty}{\usepackage{upquote}}{}
\begin{document}

\maketitle
\newpage

1.1VECTORES NUMERICOS
Crear un vector numerico vacio y a�adirle luego sus elementos
\begin{knitrout}
\definecolor{shadecolor}{rgb}{0.969, 0.969, 0.969}\color{fgcolor}\begin{kframe}
\begin{alltt}
\hlstd{v} \hlkwb{<-} \hlkwd{numeric}\hlstd{(}\hlnum{3}\hlstd{);v}
\end{alltt}
\begin{verbatim}
## [1] 0 0 0
\end{verbatim}
\begin{alltt}
\hlstd{v[}\hlnum{3}\hlstd{]} \hlkwb{<-} \hlnum{17}\hlstd{; v}
\end{alltt}
\begin{verbatim}
## [1]  0  0 17
\end{verbatim}
\end{kframe}
\end{knitrout}

Crear un vector num�rico asign�ndole todos sus elementos o valores
\begin{knitrout}
\definecolor{shadecolor}{rgb}{0.969, 0.969, 0.969}\color{fgcolor}\begin{kframe}
\begin{alltt}
\hlstd{x} \hlkwb{<-} \hlkwd{c}\hlstd{(}\hlnum{2}\hlstd{,} \hlnum{4}\hlstd{,} \hlnum{3.1}\hlstd{,} \hlnum{8}\hlstd{,} \hlnum{6}\hlstd{)}
\hlstd{x}\hlkwb{<-} \hlkwd{edit}\hlstd{(x)}
\hlstd{x}
\end{alltt}
\begin{verbatim}
## [1] 2.0 4.0 3.1 8.0 6.0
\end{verbatim}
\begin{alltt}
\hlkwd{is.integer}\hlstd{(x)}
\end{alltt}
\begin{verbatim}
## [1] FALSE
\end{verbatim}
\begin{alltt}
\hlkwd{is.double}\hlstd{(x)}
\end{alltt}
\begin{verbatim}
## [1] TRUE
\end{verbatim}
\begin{alltt}
\hlkwd{length}\hlstd{(x)}
\end{alltt}
\begin{verbatim}
## [1] 5
\end{verbatim}
\end{kframe}
\end{knitrout}

Crear un vector num�rico dando un rango de valores
\begin{knitrout}
\definecolor{shadecolor}{rgb}{0.969, 0.969, 0.969}\color{fgcolor}\begin{kframe}
\begin{alltt}
\hlstd{y} \hlkwb{=} \hlnum{1}\hlopt{:}\hlnum{4}\hlstd{; y}
\end{alltt}
\begin{verbatim}
## [1] 1 2 3 4
\end{verbatim}
\begin{alltt}
\hlstd{y[}\hlnum{2}\hlstd{]} \hlkwb{<-} \hlnum{5}
\hlstd{u} \hlkwb{<-} \hlnum{1}\hlopt{:}\hlnum{12}\hlstd{; u1}\hlkwb{=}\hlstd{u[}\hlnum{2} \hlopt{*} \hlnum{1}\hlopt{:}\hlnum{5}\hlstd{]}
\end{alltt}
\end{kframe}
\end{knitrout}

Crear un vector num�rico utilizando la funci�n assign()
\begin{knitrout}
\definecolor{shadecolor}{rgb}{0.969, 0.969, 0.969}\color{fgcolor}\begin{kframe}
\begin{alltt}
\hlkwd{assign}\hlstd{(}\hlstr{"z"}\hlstd{,} \hlkwd{c}\hlstd{(x,} \hlnum{0}\hlstd{, x)); z}
\end{alltt}
\begin{verbatim}
##  [1] 2.0 4.0 3.1 8.0 6.0 0.0 2.0 4.0 3.1 8.0 6.0
\end{verbatim}
\end{kframe}
\end{knitrout}

Crear un vector num�rico generando una sucesi�n de valores
\begin{knitrout}
\definecolor{shadecolor}{rgb}{0.969, 0.969, 0.969}\color{fgcolor}\begin{kframe}
\begin{alltt}
\hlstd{s1} \hlkwb{<-} \hlkwd{seq}\hlstd{(}\hlnum{2}\hlstd{,} \hlnum{10}\hlstd{); s1}
\end{alltt}
\begin{verbatim}
## [1]  2  3  4  5  6  7  8  9 10
\end{verbatim}
\begin{alltt}
\hlstd{s2} \hlkwb{=} \hlkwd{seq}\hlstd{(}\hlkwc{from}\hlstd{=}\hlopt{-}\hlnum{1}\hlstd{,} \hlkwc{to}\hlstd{=}\hlnum{5}\hlstd{); s2}
\end{alltt}
\begin{verbatim}
## [1] -1  0  1  2  3  4  5
\end{verbatim}
\begin{alltt}
\hlstd{s3}\hlkwb{<-}\hlkwd{seq}\hlstd{(}\hlkwc{to}\hlstd{=}\hlnum{2}\hlstd{,} \hlkwc{from}\hlstd{=}\hlopt{-}\hlnum{2}\hlstd{); s3}
\end{alltt}
\begin{verbatim}
## [1] -2 -1  0  1  2
\end{verbatim}
\begin{alltt}
\hlstd{s4}\hlkwb{=}\hlkwd{seq}\hlstd{(}\hlkwc{from}\hlstd{=}\hlopt{-}\hlnum{3}\hlstd{,} \hlkwc{to}\hlstd{=}\hlnum{3}\hlstd{,} \hlkwc{by}\hlstd{=}\hlnum{0.2}\hlstd{); s4}
\end{alltt}
\begin{verbatim}
##  [1] -3.0 -2.8 -2.6 -2.4 -2.2 -2.0 -1.8 -1.6 -1.4 -1.2 -1.0 -0.8 -0.6 -0.4
## [15] -0.2  0.0  0.2  0.4  0.6  0.8  1.0  1.2  1.4  1.6  1.8  2.0  2.2  2.4
## [29]  2.6  2.8  3.0
\end{verbatim}
\begin{alltt}
\hlstd{s5} \hlkwb{<-} \hlkwd{rep}\hlstd{(s3,} \hlkwc{times}\hlstd{=}\hlnum{3}\hlstd{); s5}
\end{alltt}
\begin{verbatim}
##  [1] -2 -1  0  1  2 -2 -1  0  1  2 -2 -1  0  1  2
\end{verbatim}
\end{kframe}
\end{knitrout}

1.1.1OPERACIONES CON VECTORES NUMERICOS 
\begin{knitrout}
\definecolor{shadecolor}{rgb}{0.969, 0.969, 0.969}\color{fgcolor}\begin{kframe}
\begin{alltt}
\hlstd{x}\hlkwb{=}\hlnum{8}
\hlnum{1}\hlopt{/}\hlstd{x}
\end{alltt}
\begin{verbatim}
## [1] 0.125
\end{verbatim}
\begin{alltt}
\hlstd{v}\hlkwb{=}\hlnum{2}\hlopt{*}\hlstd{x}\hlopt{+}\hlstd{z}\hlopt{+}\hlnum{1}\hlstd{; v}
\end{alltt}
\begin{verbatim}
##  [1] 19.0 21.0 20.1 25.0 23.0 17.0 19.0 21.0 20.1 25.0 23.0
\end{verbatim}
\begin{alltt}
\hlstd{e1} \hlkwb{<-} \hlkwd{c}\hlstd{(}\hlnum{1}\hlstd{,} \hlnum{2}\hlstd{,} \hlnum{3}\hlstd{,} \hlnum{4}\hlstd{); e2}\hlkwb{<-}\hlkwd{c}\hlstd{(}\hlnum{4}\hlstd{,} \hlnum{5}\hlstd{,} \hlnum{6}\hlstd{,} \hlnum{7}\hlstd{);} \hlkwd{crossprod}\hlstd{(e1, e2)}
\end{alltt}
\begin{verbatim}
##      [,1]
## [1,]   60
\end{verbatim}
\begin{alltt}
\hlkwd{t}\hlstd{(e1)}\hlopt\hlstd{e2}
\end{alltt}
\begin{verbatim}
##      [,1]
## [1,]   60
\end{verbatim}
\end{kframe}
\end{knitrout}

OPERACIONES DE FUNCIONES SOBRE VECTORES NUMERICOS
\begin{knitrout}
\definecolor{shadecolor}{rgb}{0.969, 0.969, 0.969}\color{fgcolor}\begin{kframe}
\begin{alltt}
\hlstd{xt}\hlkwb{<-} \hlkwd{t}\hlstd{(x); xt}
\end{alltt}
\begin{verbatim}
##      [,1]
## [1,]    8
\end{verbatim}
\begin{alltt}
\hlstd{u} \hlkwb{=} \hlkwd{exp}\hlstd{(y);u}
\end{alltt}
\begin{verbatim}
## [1]   2.718282 148.413159  20.085537  54.598150
\end{verbatim}
\begin{alltt}
\hlkwd{options}\hlstd{(}\hlkwc{digits}\hlstd{=}\hlnum{10}\hlstd{); u}
\end{alltt}
\begin{verbatim}
## [1]   2.718281828 148.413159103  20.085536923  54.598150033
\end{verbatim}
\end{kframe}
\end{knitrout}

OTRAS OPERACIONES
\begin{knitrout}
\definecolor{shadecolor}{rgb}{0.969, 0.969, 0.969}\color{fgcolor}\begin{kframe}
\begin{alltt}
\hlstd{resum} \hlkwb{<-} \hlkwd{c}\hlstd{(}\hlkwd{length}\hlstd{(y),} \hlkwd{sum}\hlstd{(y),} \hlkwd{prod}\hlstd{(y),} \hlkwd{min}\hlstd{(y),} \hlkwd{max}\hlstd{(y)); resum}
\end{alltt}
\begin{verbatim}
## [1]  4 13 60  1  5
\end{verbatim}
\begin{alltt}
\hlstd{yo} \hlkwb{<-} \hlkwd{sort}\hlstd{(y); yo}
\end{alltt}
\begin{verbatim}
## [1] 1 3 4 5
\end{verbatim}
\end{kframe}
\end{knitrout}

1.2 VECTORES DE CARACTERES
\begin{knitrout}
\definecolor{shadecolor}{rgb}{0.969, 0.969, 0.969}\color{fgcolor}\begin{kframe}
\begin{alltt}
\hlstd{S}\hlkwb{<-}\hlkwd{character}\hlstd{()}
\hlstd{S}
\end{alltt}
\begin{verbatim}
## character(0)
\end{verbatim}
\end{kframe}
\end{knitrout}


FORMA 4-Crear una matriz a partir de la uni�n de vectores 
\begin{knitrout}
\definecolor{shadecolor}{rgb}{0.969, 0.969, 0.969}\color{fgcolor}\begin{kframe}
\begin{alltt}
\hlstd{x1} \hlkwb{<-} \hlkwd{seq}\hlstd{(}\hlnum{0}\hlstd{,} \hlnum{10}\hlstd{,} \hlnum{2}\hlstd{); x1}
\end{alltt}
\begin{verbatim}
## [1]  0  2  4  6  8 10
\end{verbatim}
\begin{alltt}
\hlstd{x2} \hlkwb{<-} \hlkwd{seq}\hlstd{(}\hlnum{1}\hlstd{,} \hlnum{11}\hlstd{,} \hlnum{2}\hlstd{); x2}
\end{alltt}
\begin{verbatim}
## [1]  1  3  5  7  9 11
\end{verbatim}
\begin{alltt}
\hlstd{x3} \hlkwb{<-} \hlkwd{runif}\hlstd{(}\hlnum{6}\hlstd{); x3}
\end{alltt}
\begin{verbatim}
## [1] 0.49822225748 0.06834137882 0.01149970759 0.48312492459 0.46736010094
## [6] 0.49631722132
\end{verbatim}
\begin{alltt}
\hlstd{Xcol} \hlkwb{<-} \hlkwd{cbind}\hlstd{(x1, x2, x3); Xcol}
\end{alltt}
\begin{verbatim}
##      x1 x2            x3
## [1,]  0  1 0.49822225748
## [2,]  2  3 0.06834137882
## [3,]  4  5 0.01149970759
## [4,]  6  7 0.48312492459
## [5,]  8  9 0.46736010094
## [6,] 10 11 0.49631722132
\end{verbatim}
\begin{alltt}
\hlstd{Xfil} \hlkwb{<-} \hlkwd{rbind}\hlstd{(x1, x2, x3); Xfil}
\end{alltt}
\begin{verbatim}
##            [,1]          [,2]          [,3]         [,4]         [,5]
## x1 0.0000000000 2.00000000000 4.00000000000 6.0000000000 8.0000000000
## x2 1.0000000000 3.00000000000 5.00000000000 7.0000000000 9.0000000000
## x3 0.4982222575 0.06834137882 0.01149970759 0.4831249246 0.4673601009
##             [,6]
## x1 10.0000000000
## x2 11.0000000000
## x3  0.4963172213
\end{verbatim}
\begin{alltt}
\hlstd{X} \hlkwb{<-} \hlstd{Xfil[}\hlnum{1}\hlopt{:}\hlnum{3}\hlstd{,} \hlkwd{c}\hlstd{(}\hlnum{2}\hlstd{,} \hlnum{3}\hlstd{)]; X}
\end{alltt}
\begin{verbatim}
##             [,1]          [,2]
## x1 2.00000000000 4.00000000000
## x2 3.00000000000 5.00000000000
## x3 0.06834137882 0.01149970759
\end{verbatim}
\end{kframe}
\end{knitrout}

FORMA 2-Crear un vector de caracteres asign�ndole todos sus elementos
\begin{knitrout}
\definecolor{shadecolor}{rgb}{0.969, 0.969, 0.969}\color{fgcolor}\begin{kframe}
\begin{alltt}
\hlstd{deptos} \hlkwb{<-} \hlkwd{c}\hlstd{(}\hlstr{"Santa Ana"}\hlstd{,} \hlstr{"Sonsonate"}\hlstd{,} \hlstr{"San Salvador"}\hlstd{); deptos}
\end{alltt}
\begin{verbatim}
## [1] "Santa Ana"    "Sonsonate"    "San Salvador"
\end{verbatim}
\begin{alltt}
\hlstd{deptos[}\hlnum{4}\hlstd{]}\hlkwb{=}\hlstr{"Ahuachapán"}\hlstd{; deptos}
\end{alltt}
\begin{verbatim}
## [1] "Santa Ana"    "Sonsonate"    "San Salvador" "Ahuachapán"
\end{verbatim}
\end{kframe}
\end{knitrout}




2.CREACI�N Y MANEJO DE MATRICES.
2.1CREACI�N DE MATRICES NUM�RICAS.

FORMA 1-Crear una matriz num�rica vac�a y a�adirle luego sus elementos. 
\begin{knitrout}
\definecolor{shadecolor}{rgb}{0.969, 0.969, 0.969}\color{fgcolor}\begin{kframe}
\begin{alltt}
\hlstd{M} \hlkwb{<-} \hlkwd{matrix}\hlstd{(}\hlkwd{numeric}\hlstd{(),} \hlkwc{nrow} \hlstd{=} \hlnum{3}\hlstd{,} \hlkwc{ncol}\hlstd{=}\hlnum{4}\hlstd{);M}
\end{alltt}
\begin{verbatim}
##      [,1] [,2] [,3] [,4]
## [1,]   NA   NA   NA   NA
## [2,]   NA   NA   NA   NA
## [3,]   NA   NA   NA   NA
\end{verbatim}
\begin{alltt}
\hlstd{M[}\hlnum{2}\hlstd{,}\hlnum{3}\hlstd{]} \hlkwb{<-} \hlnum{6}\hlstd{; M}
\end{alltt}
\begin{verbatim}
##      [,1] [,2] [,3] [,4]
## [1,]   NA   NA   NA   NA
## [2,]   NA   NA    6   NA
## [3,]   NA   NA   NA   NA
\end{verbatim}
\end{kframe}
\end{knitrout}

FORMA 2-Crear una matriz num�rica asign�ndole todos sus elementos o valores. 
\begin{knitrout}
\definecolor{shadecolor}{rgb}{0.969, 0.969, 0.969}\color{fgcolor}\begin{kframe}
\begin{alltt}
\hlstd{A} \hlkwb{<-} \hlkwd{matrix}\hlstd{(}\hlkwd{c}\hlstd{(}\hlnum{2}\hlstd{,} \hlnum{4}\hlstd{,} \hlnum{6}\hlstd{,} \hlnum{8}\hlstd{,} \hlnum{10}\hlstd{,} \hlnum{12}\hlstd{),} \hlkwc{nrow}\hlstd{=}\hlnum{2}\hlstd{,} \hlkwc{ncol}\hlstd{=}\hlnum{3}\hlstd{); A}
\end{alltt}
\begin{verbatim}
##      [,1] [,2] [,3]
## [1,]    2    6   10
## [2,]    4    8   12
\end{verbatim}
\begin{alltt}
\hlkwd{mode}\hlstd{(A)}
\end{alltt}
\begin{verbatim}
## [1] "numeric"
\end{verbatim}
\begin{alltt}
\hlkwd{dim}\hlstd{(A)}
\end{alltt}
\begin{verbatim}
## [1] 2 3
\end{verbatim}
\begin{alltt}
\hlkwd{attributes}\hlstd{(A)}
\end{alltt}
\begin{verbatim}
## $dim
## [1] 2 3
\end{verbatim}
\begin{alltt}
\hlkwd{is.matrix}\hlstd{(A)}
\end{alltt}
\begin{verbatim}
## [1] TRUE
\end{verbatim}
\begin{alltt}
\hlkwd{is.array}\hlstd{(A)}
\end{alltt}
\begin{verbatim}
## [1] TRUE
\end{verbatim}
\end{kframe}
\end{knitrout}

FORMA 3-Crear una matriz num�rica dando un rango de valores 
\begin{knitrout}
\definecolor{shadecolor}{rgb}{0.969, 0.969, 0.969}\color{fgcolor}\begin{kframe}
\begin{alltt}
\hlstd{B} \hlkwb{<-} \hlkwd{matrix}\hlstd{(}\hlnum{1}\hlopt{:}\hlnum{12}\hlstd{,} \hlkwc{nrow}\hlstd{=}\hlnum{3}\hlstd{,} \hlkwc{ncol}\hlstd{=}\hlnum{4}\hlstd{); B}
\end{alltt}
\begin{verbatim}
##      [,1] [,2] [,3] [,4]
## [1,]    1    4    7   10
## [2,]    2    5    8   11
## [3,]    3    6    9   12
\end{verbatim}
\end{kframe}
\end{knitrout}


2.2OPERACIONES CON MATRICES NUM�RICAS. 
MULTIPLICACION DE MATRICES MATRICES NUMERICAS:
\begin{knitrout}
\definecolor{shadecolor}{rgb}{0.969, 0.969, 0.969}\color{fgcolor}\begin{kframe}
\begin{alltt}
\hlstd{v}\hlkwb{<-}\hlkwd{c}\hlstd{(}\hlnum{1}\hlstd{,} \hlnum{2}\hlstd{); v} \hlopt\hlstd{A}
\end{alltt}
\begin{verbatim}
##      [,1] [,2] [,3]
## [1,]   10   22   34
\end{verbatim}
\begin{alltt}
\hlstd{P}\hlkwb{<-} \hlstd{A} \hlopt \hlstd{B; P}
\end{alltt}
\begin{verbatim}
##      [,1] [,2] [,3] [,4]
## [1,]   44   98  152  206
## [2,]   56  128  200  272
\end{verbatim}
\begin{alltt}
\hlnum{2}\hlopt{*}\hlstd{A}
\end{alltt}
\begin{verbatim}
##      [,1] [,2] [,3]
## [1,]    4   12   20
## [2,]    8   16   24
\end{verbatim}
\end{kframe}
\end{knitrout}

OPERACIONES DE FUNCIONES SOBRE MATRICES NUM�RICAS:
\begin{knitrout}
\definecolor{shadecolor}{rgb}{0.969, 0.969, 0.969}\color{fgcolor}\begin{kframe}
\begin{alltt}
\hlkwd{length}\hlstd{(A)}
\end{alltt}
\begin{verbatim}
## [1] 6
\end{verbatim}
\begin{alltt}
\hlstd{T}\hlkwb{=}\hlkwd{sqrt}\hlstd{(B); T}
\end{alltt}
\begin{verbatim}
##             [,1]        [,2]        [,3]        [,4]
## [1,] 1.000000000 2.000000000 2.645751311 3.162277660
## [2,] 1.414213562 2.236067977 2.828427125 3.316624790
## [3,] 1.732050808 2.449489743 3.000000000 3.464101615
\end{verbatim}
\begin{alltt}
\hlkwd{t}\hlstd{(A)}
\end{alltt}
\begin{verbatim}
##      [,1] [,2]
## [1,]    2    4
## [2,]    6    8
## [3,]   10   12
\end{verbatim}
\begin{alltt}
\hlstd{C} \hlkwb{<-} \hlkwd{matrix}\hlstd{(}\hlkwd{c}\hlstd{(}\hlnum{2}\hlstd{,} \hlnum{1}\hlstd{,} \hlnum{10}\hlstd{,} \hlnum{12}\hlstd{),} \hlkwc{nrow}\hlstd{=}\hlnum{2}\hlstd{,} \hlkwc{ncol}\hlstd{=}\hlnum{2}\hlstd{); C}
\end{alltt}
\begin{verbatim}
##      [,1] [,2]
## [1,]    2   10
## [2,]    1   12
\end{verbatim}
\begin{alltt}
\hlkwd{det}\hlstd{(C)}
\end{alltt}
\begin{verbatim}
## [1] 14
\end{verbatim}
\begin{alltt}
\hlstd{InvC} \hlkwb{<-} \hlkwd{solve}\hlstd{(C) ; InvC}
\end{alltt}
\begin{verbatim}
##                [,1]          [,2]
## [1,]  0.85714285714 -0.7142857143
## [2,] -0.07142857143  0.1428571429
\end{verbatim}
\begin{alltt}
\hlstd{b}\hlkwb{=}\hlkwd{diag}\hlstd{(}\hlnum{2}\hlstd{); InvC}\hlkwb{<-}\hlkwd{solve}\hlstd{(C, b); InvC}
\end{alltt}
\begin{verbatim}
##                [,1]          [,2]
## [1,]  0.85714285714 -0.7142857143
## [2,] -0.07142857143  0.1428571429
\end{verbatim}
\begin{alltt}
\hlkwd{eigen}\hlstd{(C)}
\end{alltt}
\begin{verbatim}
## $values
## [1] 12.916079783  1.083920217
## 
## $vectors
##               [,1]           [,2]
## [1,] -0.6754894393 -0.99583021557
## [2,] -0.7373696613  0.09122599279
\end{verbatim}
\begin{alltt}
\hlkwd{diag}\hlstd{(C)}
\end{alltt}
\begin{verbatim}
## [1]  2 12
\end{verbatim}
\begin{alltt}
\hlkwd{diag}\hlstd{(u1)}
\end{alltt}
\begin{verbatim}
##      [,1] [,2] [,3] [,4] [,5]
## [1,]    2    0    0    0    0
## [2,]    0    4    0    0    0
## [3,]    0    0    6    0    0
## [4,]    0    0    0    8    0
## [5,]    0    0    0    0   10
\end{verbatim}
\begin{alltt}
\hlkwd{diag}\hlstd{(}\hlnum{3}\hlstd{)}
\end{alltt}
\begin{verbatim}
##      [,1] [,2] [,3]
## [1,]    1    0    0
## [2,]    0    1    0
## [3,]    0    0    1
\end{verbatim}
\end{kframe}
\end{knitrout}


OTRAS OPERACIONES: 
\begin{knitrout}
\definecolor{shadecolor}{rgb}{0.969, 0.969, 0.969}\color{fgcolor}\begin{kframe}
\begin{alltt}
\hlkwd{c}\hlstd{(}\hlkwd{length}\hlstd{(A),} \hlkwd{sum}\hlstd{(A),} \hlkwd{prod}\hlstd{(A),} \hlkwd{min}\hlstd{(A),} \hlkwd{max}\hlstd{(A))}
\end{alltt}
\begin{verbatim}
## [1]     6    42 46080     2    12
\end{verbatim}
\begin{alltt}
\hlstd{O} \hlkwb{<-} \hlkwd{matrix}\hlstd{(}\hlkwd{sort}\hlstd{(C),} \hlkwc{nrow}\hlstd{=}\hlnum{2}\hlstd{,} \hlkwc{ncol}\hlstd{=}\hlnum{2}\hlstd{); O}
\end{alltt}
\begin{verbatim}
##      [,1] [,2]
## [1,]    1   10
## [2,]    2   12
\end{verbatim}
\end{kframe}
\end{knitrout}

2.3     CREACI�N DE UNA MATRIZ DE CADENAS
\begin{knitrout}
\definecolor{shadecolor}{rgb}{0.969, 0.969, 0.969}\color{fgcolor}\begin{kframe}
\begin{alltt}
\hlstd{nombres} \hlkwb{<-} \hlkwd{matrix}\hlstd{(}\hlkwd{c}\hlstd{(}\hlstr{"Carlos"}\hlstd{,} \hlstr{"Jos�"}\hlstd{,} \hlstr{"Ana"}\hlstd{,} \hlstr{"Ren�"}\hlstd{,} \hlstr{"Maria"}\hlstd{,} \hlstr{"Mario"}\hlstd{),} \hlkwc{nrow}\hlstd{=}\hlnum{3}\hlstd{,} \hlkwc{ncol}\hlstd{=}\hlnum{2}\hlstd{)}
\hlstd{nombres}
\end{alltt}
\begin{verbatim}
##      [,1]     [,2]   
## [1,] "Carlos" "Ren�" 
## [2,] "Jos�"   "Maria"
## [3,] "Ana"    "Mario"
\end{verbatim}
\end{kframe}
\end{knitrout}

3.CREACI�N Y MANEJO DE MATRICES INDEXADAS (ARRAY)
\begin{knitrout}
\definecolor{shadecolor}{rgb}{0.969, 0.969, 0.969}\color{fgcolor}\begin{kframe}
\begin{alltt}
\hlstd{X} \hlkwb{<-} \hlkwd{array}\hlstd{(}\hlkwd{c}\hlstd{(}\hlnum{1}\hlstd{,} \hlnum{3}\hlstd{,} \hlnum{5}\hlstd{,} \hlnum{7}\hlstd{,} \hlnum{9}\hlstd{,} \hlnum{11}\hlstd{),} \hlkwc{dim}\hlstd{=}\hlkwd{c}\hlstd{(}\hlnum{2}\hlstd{,} \hlnum{3}\hlstd{)); X}
\end{alltt}
\begin{verbatim}
##      [,1] [,2] [,3]
## [1,]    1    5    9
## [2,]    3    7   11
\end{verbatim}
\begin{alltt}
\hlstd{Z} \hlkwb{<-} \hlkwd{array}\hlstd{(}\hlnum{1}\hlstd{,} \hlkwd{c}\hlstd{(}\hlnum{3}\hlstd{,} \hlnum{3}\hlstd{)); Z}
\end{alltt}
\begin{verbatim}
##      [,1] [,2] [,3]
## [1,]    1    1    1
## [2,]    1    1    1
## [3,]    1    1    1
\end{verbatim}
\begin{alltt}
\hlstd{W} \hlkwb{<-} \hlnum{2}\hlopt{*}\hlstd{Z}\hlopt{+}\hlnum{1}\hlstd{; W}
\end{alltt}
\begin{verbatim}
##      [,1] [,2] [,3]
## [1,]    3    3    3
## [2,]    3    3    3
## [3,]    3    3    3
\end{verbatim}
\begin{alltt}
\hlstd{TX} \hlkwb{<-} \hlkwd{t}\hlstd{(X); TX}
\end{alltt}
\begin{verbatim}
##      [,1] [,2]
## [1,]    1    3
## [2,]    5    7
## [3,]    9   11
\end{verbatim}
\begin{alltt}
\hlstd{a} \hlkwb{<-} \hlkwd{c}\hlstd{(}\hlnum{2}\hlstd{,} \hlnum{4}\hlstd{,} \hlnum{6}\hlstd{); a}
\end{alltt}
\begin{verbatim}
## [1] 2 4 6
\end{verbatim}
\begin{alltt}
\hlstd{b} \hlkwb{<-} \hlnum{1}\hlopt{:}\hlnum{3}\hlstd{;b}
\end{alltt}
\begin{verbatim}
## [1] 1 2 3
\end{verbatim}
\begin{alltt}
\hlstd{M} \hlkwb{<-} \hlstd{a} \hlopt \hlstd{b; M}
\end{alltt}
\begin{verbatim}
##      [,1] [,2] [,3]
## [1,]    2    4    6
## [2,]    4    8   12
## [3,]    6   12   18
\end{verbatim}
\begin{alltt}
\hlstd{c} \hlkwb{<-} \hlstd{a} \hlopt{*} \hlstd{b; c}
\end{alltt}
\begin{verbatim}
## [1]  2  8 18
\end{verbatim}
\begin{alltt}
\hlstd{Arreglo3} \hlkwb{<-} \hlkwd{array}\hlstd{(}\hlkwd{c}\hlstd{(}\hlnum{1}\hlopt{:}\hlnum{8}\hlstd{,} \hlnum{11}\hlopt{:}\hlnum{18}\hlstd{,} \hlnum{111}\hlopt{:}\hlnum{118}\hlstd{),} \hlkwc{dim} \hlstd{=} \hlkwd{c}\hlstd{(}\hlnum{2}\hlstd{,} \hlnum{4}\hlstd{,} \hlnum{3}\hlstd{))}
\hlstd{Arreglo3}
\end{alltt}
\begin{verbatim}
## , , 1
## 
##      [,1] [,2] [,3] [,4]
## [1,]    1    3    5    7
## [2,]    2    4    6    8
## 
## , , 2
## 
##      [,1] [,2] [,3] [,4]
## [1,]   11   13   15   17
## [2,]   12   14   16   18
## 
## , , 3
## 
##      [,1] [,2] [,3] [,4]
## [1,]  111  113  115  117
## [2,]  112  114  116  118
\end{verbatim}
\end{kframe}
\end{knitrout}

\end{document}

