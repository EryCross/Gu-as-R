\documentclass[10pt,a4paper]{article}\usepackage[]{graphicx}\usepackage[]{color}
%% maxwidth is the original width if it is less than linewidth
%% otherwise use linewidth (to make sure the graphics do not exceed the margin)
\makeatletter
\def\maxwidth{ %
  \ifdim\Gin@nat@width>\linewidth
    \linewidth
  \else
    \Gin@nat@width
  \fi
}
\makeatother

\definecolor{fgcolor}{rgb}{0.345, 0.345, 0.345}
\newcommand{\hlnum}[1]{\textcolor[rgb]{0.686,0.059,0.569}{#1}}%
\newcommand{\hlstr}[1]{\textcolor[rgb]{0.192,0.494,0.8}{#1}}%
\newcommand{\hlcom}[1]{\textcolor[rgb]{0.678,0.584,0.686}{\textit{#1}}}%
\newcommand{\hlopt}[1]{\textcolor[rgb]{0,0,0}{#1}}%
\newcommand{\hlstd}[1]{\textcolor[rgb]{0.345,0.345,0.345}{#1}}%
\newcommand{\hlkwa}[1]{\textcolor[rgb]{0.161,0.373,0.58}{\textbf{#1}}}%
\newcommand{\hlkwb}[1]{\textcolor[rgb]{0.69,0.353,0.396}{#1}}%
\newcommand{\hlkwc}[1]{\textcolor[rgb]{0.333,0.667,0.333}{#1}}%
\newcommand{\hlkwd}[1]{\textcolor[rgb]{0.737,0.353,0.396}{\textbf{#1}}}%

\usepackage{framed}
\makeatletter
\newenvironment{kframe}{%
 \def\at@end@of@kframe{}%
 \ifinner\ifhmode%
  \def\at@end@of@kframe{\end{minipage}}%
  \begin{minipage}{\columnwidth}%
 \fi\fi%
 \def\FrameCommand##1{\hskip\@totalleftmargin \hskip-\fboxsep
 \colorbox{shadecolor}{##1}\hskip-\fboxsep
     % There is no \\@totalrightmargin, so:
     \hskip-\linewidth \hskip-\@totalleftmargin \hskip\columnwidth}%
 \MakeFramed {\advance\hsize-\width
   \@totalleftmargin\z@ \linewidth\hsize
   \@setminipage}}%
 {\par\unskip\endMakeFramed%
 \at@end@of@kframe}
\makeatother

\definecolor{shadecolor}{rgb}{.97, .97, .97}
\definecolor{messagecolor}{rgb}{0, 0, 0}
\definecolor{warningcolor}{rgb}{1, 0, 1}
\definecolor{errorcolor}{rgb}{1, 0, 0}
\newenvironment{knitrout}{}{} % an empty environment to be redefined in TeX

\usepackage{alltt}
\usepackage[latin1]{inputenc}
\usepackage{amsmath}
\usepackage{amsfonts}
\usepackage{amssymb}
\author{Erika Martínez}
\title{Guías prácticas}
\IfFileExists{upquote.sty}{\usepackage{upquote}}{}
\begin{document}

\maketitle
\newpage

UNIDAD 6: Pr?ctica 25 - Dise?os por bloques

EJEMPLO 1. 
\begin{knitrout}
\definecolor{shadecolor}{rgb}{0.969, 0.969, 0.969}\color{fgcolor}\begin{kframe}
\begin{alltt}
\hlcom{#Se probaran 5 raciones respecto a sus diferencias en el engorde de novillos. Se dispone de 20 novillos }
\hlcom{#para el experimento, que se distribuyen en 4 bloques (5 novillos por bloque) con base a sus pesos, al }
\hlcom{#iniciar la prueba de engorde, los novillos m?s pesados se agruparon en un bloque, en otro se }
\hlcom{#agruparon los 5 siguientes m?s pesados y as? sucesivamente. Los 5 tratamientos (raciones) se }
\hlcom{#asignaron al azar dentro de cada bloque.}

\hlcom{#Utilizando un nivel de significancia del 5%, contraste la hip?tesis de que las cinco }
\hlcom{#raciones de comida producen el mismo efecto de engorde en los novillos.}

\hlcom{# Definiendo el vector que contendr? el bloque al cual pertenecen los novillos.}
\hlstd{bloques} \hlkwb{<-} \hlkwd{gl}\hlstd{(}\hlkwc{n}\hlstd{=}\hlnum{4}\hlstd{,} \hlkwc{k}\hlstd{=}\hlnum{1}\hlstd{,} \hlkwc{length}\hlstd{=}\hlnum{20}\hlstd{);bloques}
\end{alltt}
\begin{verbatim}
##  [1] 1 2 3 4 1 2 3 4 1 2 3 4 1 2 3 4 1 2 3 4
## Levels: 1 2 3 4
\end{verbatim}
\begin{alltt}
\hlcom{# Se crea el vector que contendr? los tratamientos de los novillos (raciones de alimento)}
\hlstd{tratamientos} \hlkwb{<-} \hlkwd{gl}\hlstd{(}\hlkwc{n}\hlstd{=}\hlnum{5}\hlstd{,} \hlkwc{k}\hlstd{=}\hlnum{4}\hlstd{);tratamientos}
\end{alltt}
\begin{verbatim}
##  [1] 1 1 1 1 2 2 2 2 3 3 3 3 4 4 4 4 5 5 5 5
## Levels: 1 2 3 4 5
\end{verbatim}
\begin{alltt}
\hlcom{# Se digitan los pesos de los novillos }
\hlstd{peso} \hlkwb{<-} \hlkwd{c}\hlstd{(}\hlnum{0.9}\hlstd{,}\hlnum{1.4}\hlstd{,}\hlnum{1.4}\hlstd{,}\hlnum{2.3}\hlstd{,}\hlnum{3.6}\hlstd{,}\hlnum{3.2}\hlstd{,}\hlnum{4.5}\hlstd{,}\hlnum{4.1}\hlstd{,}\hlnum{0.5}\hlstd{,}\hlnum{0.9}\hlstd{,}\hlnum{0.5}\hlstd{,}\hlnum{0.9}\hlstd{,}\hlnum{3.6}\hlstd{,}\hlnum{3.6}\hlstd{,}\hlnum{3.2}\hlstd{,}\hlnum{3.6}\hlstd{,}\hlnum{1.8}\hlstd{,}\hlnum{1.8}\hlstd{,}\hlnum{0.9}\hlstd{,}\hlnum{1.4} \hlstd{);peso}
\end{alltt}
\begin{verbatim}
##  [1] 0.9 1.4 1.4 2.3 3.6 3.2 4.5 4.1 0.5 0.9 0.5 0.9 3.6 3.6 3.2 3.6 1.8
## [18] 1.8 0.9 1.4
\end{verbatim}
\begin{alltt}
\hlcom{# Se registra en una hoja de datos los resultados del experimento}
\hlstd{datos2} \hlkwb{<-} \hlkwd{data.frame}\hlstd{(}\hlkwc{bloques} \hlstd{= bloques,} \hlkwc{tratamientos} \hlstd{= tratamientos,} \hlkwc{peso} \hlstd{= peso);datos2}
\end{alltt}
\begin{verbatim}
##    bloques tratamientos peso
## 1        1            1  0.9
## 2        2            1  1.4
## 3        3            1  1.4
## 4        4            1  2.3
## 5        1            2  3.6
## 6        2            2  3.2
## 7        3            2  4.5
## 8        4            2  4.1
## 9        1            3  0.5
## 10       2            3  0.9
## 11       3            3  0.5
## 12       4            3  0.9
## 13       1            4  3.6
## 14       2            4  3.6
## 15       3            4  3.2
## 16       4            4  3.6
## 17       1            5  1.8
## 18       2            5  1.8
## 19       3            5  0.9
## 20       4            5  1.4
\end{verbatim}
\begin{alltt}
\hlcom{# Se aplica el an?lisis de varianza }
\hlstd{mod2} \hlkwb{<-} \hlkwd{aov}\hlstd{(peso} \hlopt{~} \hlstd{tratamientos} \hlopt{+} \hlstd{bloques,} \hlkwc{data} \hlstd{= datos2)}

\hlcom{# Se muestra la tabla ANOVA del experimento }
\hlkwd{summary}\hlstd{(mod2)}
\end{alltt}
\begin{verbatim}
##              Df Sum Sq Mean Sq F value  Pr(>F)    
## tratamientos  4 30.712   7.678  39.107 8.6e-07 ***
## bloques       3  0.462   0.154   0.784   0.526    
## Residuals    12  2.356   0.196                    
## ---
## Signif. codes:  0 '***' 0.001 '**' 0.01 '*' 0.05 '.' 0.1 ' ' 1
\end{verbatim}
\end{kframe}
\end{knitrout}



\end{document}
