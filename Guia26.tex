\documentclass[10pt,a4paper]{article}\usepackage[]{graphicx}\usepackage[]{color}
%% maxwidth is the original width if it is less than linewidth
%% otherwise use linewidth (to make sure the graphics do not exceed the margin)
\makeatletter
\def\maxwidth{ %
  \ifdim\Gin@nat@width>\linewidth
    \linewidth
  \else
    \Gin@nat@width
  \fi
}
\makeatother

\definecolor{fgcolor}{rgb}{0.345, 0.345, 0.345}
\newcommand{\hlnum}[1]{\textcolor[rgb]{0.686,0.059,0.569}{#1}}%
\newcommand{\hlstr}[1]{\textcolor[rgb]{0.192,0.494,0.8}{#1}}%
\newcommand{\hlcom}[1]{\textcolor[rgb]{0.678,0.584,0.686}{\textit{#1}}}%
\newcommand{\hlopt}[1]{\textcolor[rgb]{0,0,0}{#1}}%
\newcommand{\hlstd}[1]{\textcolor[rgb]{0.345,0.345,0.345}{#1}}%
\newcommand{\hlkwa}[1]{\textcolor[rgb]{0.161,0.373,0.58}{\textbf{#1}}}%
\newcommand{\hlkwb}[1]{\textcolor[rgb]{0.69,0.353,0.396}{#1}}%
\newcommand{\hlkwc}[1]{\textcolor[rgb]{0.333,0.667,0.333}{#1}}%
\newcommand{\hlkwd}[1]{\textcolor[rgb]{0.737,0.353,0.396}{\textbf{#1}}}%

\usepackage{framed}
\makeatletter
\newenvironment{kframe}{%
 \def\at@end@of@kframe{}%
 \ifinner\ifhmode%
  \def\at@end@of@kframe{\end{minipage}}%
  \begin{minipage}{\columnwidth}%
 \fi\fi%
 \def\FrameCommand##1{\hskip\@totalleftmargin \hskip-\fboxsep
 \colorbox{shadecolor}{##1}\hskip-\fboxsep
     % There is no \\@totalrightmargin, so:
     \hskip-\linewidth \hskip-\@totalleftmargin \hskip\columnwidth}%
 \MakeFramed {\advance\hsize-\width
   \@totalleftmargin\z@ \linewidth\hsize
   \@setminipage}}%
 {\par\unskip\endMakeFramed%
 \at@end@of@kframe}
\makeatother

\definecolor{shadecolor}{rgb}{.97, .97, .97}
\definecolor{messagecolor}{rgb}{0, 0, 0}
\definecolor{warningcolor}{rgb}{1, 0, 1}
\definecolor{errorcolor}{rgb}{1, 0, 0}
\newenvironment{knitrout}{}{} % an empty environment to be redefined in TeX

\usepackage{alltt}
\usepackage[latin1]{inputenc}
\usepackage{amsmath}
\usepackage{amsfonts}
\usepackage{amssymb}
\author{Erika Martínez}
\title{Guías prácticas}
\IfFileExists{upquote.sty}{\usepackage{upquote}}{}
\begin{document}

\maketitle
\newpage

UNIDAD 6: Pr?ctica 26 - Dise?os bifactoriales

EJEMPLO 1. 
\begin{knitrout}
\definecolor{shadecolor}{rgb}{0.969, 0.969, 0.969}\color{fgcolor}\begin{kframe}
\begin{alltt}
\hlcom{#Se llev? a cabo un estudio del efecto de la temperatura sobre el porcentaje de encogimiento}
\hlcom{#de telas te?idas, con dos r?plicas para cada uno de cuatro tipos de tela en un dise?o totalmente }
\hlcom{#aleatorizado. Los datos son el porcentaje de encogimiento de dos r?plicas de tela }
\hlcom{#secadas a cuatro temperaturas.}

\hlcom{#Utilizando un nivel de significancia del 5%}
\hlcom{# Definiendo el vector que contendr? el factor A.}
\hlstd{FactorA} \hlkwb{<-} \hlkwd{gl}\hlstd{(}\hlkwc{n}\hlstd{=}\hlnum{4}\hlstd{,} \hlkwc{k}\hlstd{=}\hlnum{8}\hlstd{,} \hlkwc{length}\hlstd{=}\hlnum{32}\hlstd{);FactorA}
\end{alltt}
\begin{verbatim}
##  [1] 1 1 1 1 1 1 1 1 2 2 2 2 2 2 2 2 3 3 3 3 3 3 3 3 4 4 4 4 4 4 4 4
## Levels: 1 2 3 4
\end{verbatim}
\begin{alltt}
\hlcom{# Se crea el vector que contendr? los tratamientos de los novillos (raciones de alimento).}
\hlstd{FactorB}\hlkwb{<-} \hlkwd{gl}\hlstd{(}\hlkwc{n}\hlstd{=}\hlnum{4}\hlstd{,} \hlkwc{k}\hlstd{=}\hlnum{2}\hlstd{,}\hlkwc{length}\hlstd{=}\hlnum{32}\hlstd{);FactorB}
\end{alltt}
\begin{verbatim}
##  [1] 1 1 2 2 3 3 4 4 1 1 2 2 3 3 4 4 1 1 2 2 3 3 4 4 1 1 2 2 3 3 4 4
## Levels: 1 2 3 4
\end{verbatim}
\begin{alltt}
\hlcom{# Se digitan los pesos de los novillos }
\hlstd{Porcentaje} \hlkwb{<-} \hlkwd{c}\hlstd{(}\hlnum{1.8}\hlstd{,} \hlnum{2.1}\hlstd{,} \hlnum{2.0}\hlstd{,} \hlnum{2.1}\hlstd{,} \hlnum{4.6}\hlstd{,} \hlnum{5.0}\hlstd{,} \hlnum{7.5}\hlstd{,} \hlnum{7.9}\hlstd{,} \hlnum{2.2}\hlstd{,} \hlnum{2.4}\hlstd{,}\hlnum{4.2}\hlstd{,} \hlnum{4.0}\hlstd{,} \hlnum{5.4}\hlstd{,} \hlnum{5.6}\hlstd{,}
\hlnum{9.8}\hlstd{,} \hlnum{9.2}\hlstd{,} \hlnum{2.8}\hlstd{,} \hlnum{3.2}\hlstd{,} \hlnum{4.4}\hlstd{,} \hlnum{4.8}\hlstd{,} \hlnum{8.7}\hlstd{,} \hlnum{8.4}\hlstd{,} \hlnum{13.2}\hlstd{,} \hlnum{13.0}\hlstd{,} \hlnum{3.2}\hlstd{,} \hlnum{3.6}\hlstd{,} \hlnum{3.3}\hlstd{,} \hlnum{3.5}\hlstd{,} \hlnum{5.7}\hlstd{,} \hlnum{5.8}\hlstd{,}
\hlnum{10.9}\hlstd{,} \hlnum{11.1}\hlstd{);Porcentaje}
\end{alltt}
\begin{verbatim}
##  [1]  1.8  2.1  2.0  2.1  4.6  5.0  7.5  7.9  2.2  2.4  4.2  4.0  5.4  5.6
## [15]  9.8  9.2  2.8  3.2  4.4  4.8  8.7  8.4 13.2 13.0  3.2  3.6  3.3  3.5
## [29]  5.7  5.8 10.9 11.1
\end{verbatim}
\begin{alltt}
\hlcom{# Se registra en una hoja de datos los resultados del experimento }
\hlstd{datos3} \hlkwb{<-} \hlkwd{data.frame}\hlstd{(}\hlkwc{FactorA} \hlstd{= FactorA,} \hlkwc{FactorB} \hlstd{= FactorB,} \hlkwc{Porcentaje}\hlstd{=Porcentaje);datos3}
\end{alltt}
\begin{verbatim}
##    FactorA FactorB Porcentaje
## 1        1       1        1.8
## 2        1       1        2.1
## 3        1       2        2.0
## 4        1       2        2.1
## 5        1       3        4.6
## 6        1       3        5.0
## 7        1       4        7.5
## 8        1       4        7.9
## 9        2       1        2.2
## 10       2       1        2.4
## 11       2       2        4.2
## 12       2       2        4.0
## 13       2       3        5.4
## 14       2       3        5.6
## 15       2       4        9.8
## 16       2       4        9.2
## 17       3       1        2.8
## 18       3       1        3.2
## 19       3       2        4.4
## 20       3       2        4.8
## 21       3       3        8.7
## 22       3       3        8.4
## 23       3       4       13.2
## 24       3       4       13.0
## 25       4       1        3.2
## 26       4       1        3.6
## 27       4       2        3.3
## 28       4       2        3.5
## 29       4       3        5.7
## 30       4       3        5.8
## 31       4       4       10.9
## 32       4       4       11.1
\end{verbatim}
\begin{alltt}
\hlcom{# Se aplica el an?lisis de varianza }
\hlstd{mod3} \hlkwb{<-} \hlkwd{aov}\hlstd{(Porcentaje} \hlopt{~} \hlstd{FactorA} \hlopt{*} \hlstd{FactorB,} \hlkwc{data} \hlstd{= datos3)}

\hlcom{# Se muestra la tabla ANOVA del experimento }
\hlkwd{summary}\hlstd{(mod3)}
\end{alltt}
\begin{verbatim}
##                 Df Sum Sq Mean Sq F value   Pr(>F)    
## FactorA          3  41.88   13.96  279.18 5.05e-14 ***
## FactorB          3 283.94   94.65 1892.91  < 2e-16 ***
## FactorA:FactorB  9  15.86    1.76   35.24 7.09e-09 ***
## Residuals       16   0.80    0.05                     
## ---
## Signif. codes:  0 '***' 0.001 '**' 0.01 '*' 0.05 '.' 0.1 ' ' 1
\end{verbatim}
\end{kframe}
\end{knitrout}


\end{document}
